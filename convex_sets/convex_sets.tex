\documentclass[12pt]{article}

\usepackage[utf8]{inputenc}
\usepackage[T1]{fontenc}
\usepackage{lmodern}
\usepackage{amsmath}
\usepackage{latexsym,amsfonts,amssymb,amsthm,amsmath}
\usepackage[shortlabels]{enumitem}
\usepackage{hyperref}
\usepackage{float}
\usepackage{graphicx}
\usepackage{subcaption}
\usepackage{booktabs}
\graphicspath{{./images/}}

\setlength{\parindent}{0in}
\setlength{\oddsidemargin}{0in}
\setlength{\textwidth}{6.5in}
\setlength{\textheight}{8.8in}
\setlength{\topmargin}{0in}
\setlength{\headheight}{18pt}

\newenvironment{Solution}
  {\begin{proof}[\textbf{Solution}]}
  {\end{proof}}

\title{Convex Sets}
\author{Kacper Kłos}

\begin{document}

\maketitle

\vspace{0.4in}

\subsection*{Exercise 2.1}
Let $C \subseteq \mathbf{R}^n$ be a convex set, with $x_1, ... , x_k \in C$, and let $\theta_1, ... , \theta_k \in R$ satisfy $\theta_i \geq 0$,
$\theta_1 + ... + \theta_k = 1$. Show that $\theta_1 x_1 + ... + \theta_k x_k \in C$. (The definition of convexity is that
this holds for $k = 2$; you must show it for arbitrary $k$.) \textit{Hint}. Use induction on $k$.
\begin{Solution}
	We know that
	\[
		\theta_1 + \cdots + \theta_{k-2} \;+\; \tilde{\theta}_{k-1} \;+\; \tilde{\theta}_{k}
		\;=\;
		1
		\quad \text{and} \quad
		\theta_1 x_1 + \cdots + \theta_{k-2} x_{k-2} \;+\; \tilde{\theta}_{k-1} \tilde{x}_{k-1} \;+\; \tilde{\theta}_{k} \tilde{x}_{k}
		\;\in\; C.
	\]
	Next, define
	\[
		\theta_{k-1}
		\;=\;
		\tilde{\theta}_{k-1} + \tilde{\theta}_{k}.
	\]
	From this we see that
	\[
		1
		\;=\;
		\frac{\tilde{\theta}_{k-1} + \tilde{\theta}_{k}}{\theta_{k-1}},
	\]
	so, using the fact that \(C\) is convex, we know that there is a \(x_{k-1}\) such that
	\[
		\tilde{\theta}_{k-1} \tilde{x}_{k-1} + \tilde{\theta}_{k} \tilde{x}_{k}
		\;=\;
		\theta_{k-1} x_{k-1}.
	\]
	Plugging this back into the original equations, we get
	\[
		\theta_1 + \cdots + \theta_{k-1} \;=\; 1
		\quad \text{and} \quad
		\theta_1 x_1 + \cdots + \theta_{k-1} x_{k-1} \;\in\; C,
	\]
	thus reducing two \(x_i\)'s into one. Repeating this procedure recursively leaves us with a single point \(x_0\), which must lie in \(C\) by its convexity.
\end{Solution}

\newpage

\subsection*{Exercise 2.2}
Show that a set is convex if and only if its intersection with any line is convex. Show that
a set is affine if and only if its intersection with any line is affine.

\begin{Solution}
	First, let us show that if a set \(C\) is convex, then its intersection with any line \(L\) is also convex.
	This follows easily because a line itself is convex, and the intersection of convex sets is convex.
	Indeed, if \(x_1, x_2 \in C \cap L\) they must belong to \(C\) and \(L\) separately, then by the convexity of \(C\), for any \(\theta \in [0,1]\),
	\[
		\theta x_1 + (1 - \theta) x_2 \in C,
	\]
	and by the convexity of \(L\),
	\[
		\theta x_1 + (1 - \theta) x_2 \in L.
	\]
	Hence,
	\[
		\theta x_1 + (1 - \theta) x_2 \in C \cap L.
	\]
	Implying convexity of $C \cap L$.

	For the other direction, let us take any two points \(x_1\) and \(x_2\) in \(C\).
	Consider the line through \(x_1\) and \(x_2\).
	Since \(C\) intersects every line in a convex set, it follows that this line, intersected with \(C\), contains all points between \(x_1\) and \(x_2\).
	In other words, for all \(\theta \in [0,1]\),
	\[
		\theta x_1 + (1 - \theta) x_2 \in C.
	\]
	Because \(x_1\) and \(x_2\) were chosen arbitrarily, it is true for all $x_1, x_2 \in C$, it follows that \(C\) is convex.

	Next, let us turn to the affine case.
	Suppose \(A\) is an affine set and let \(L\) be a line, which is also affine.
	The argument is exactly the same as in the convex case, but now we allow \(\theta \in \mathbf{R}\) (not just \([0,1]\)).
	Thus, the intersection \(A \cap L\) is affine if \(A\) is affine, and the converse holds by an identical reasoning.
\end{Solution}


\vspace{0.15in}

\subsection*{Exercise 2.5}
What is the distance between two parallel hyperplanes
\(\{x \in \mathbf{R}^n \mid a^T x = b_1\}\)
and
\(\{x \in \mathbf{R}^n \mid a^T x = b_2\}\)?

\begin{Solution}
	We know that \(a\) is parallel to the plane. Subtracting both equations, we get
	\[
		a^T (x_1 - x_2) = b_1 - b_2.
	\]
	Taking the Euclidean norm on both sides, and noting that the distance is the minimum possible value of \(\|x_1 - x_2\|_2\), we write
	\[
		\|a^T (x_1 - x_2)\|_2 = |b_1 - b_2|.
	\]
	From the standard norm inequality,
	\[
		\|a^T (x_1 - x_2)\|_2
		\;\le\;
		\|x_1 - x_2\|_2 \,\|a^T\|_2.
	\]
	The equality here represents the minimal distance. Consequently, dividing by constant \(\|a^T\|_2\)
	\[
		\min\bigl(\|x_1 - x_2\|_2\bigr)
		\;=\;
		\frac{|b_1 - b_2|}{\|a^T\|_2}.
	\]
	This value is the distance between the two hyperplanes.
\end{Solution}


\vspace{0.15in}

\subsection*{Exercise 2.7}
\textit{Voronoi description of halfspace}. Let \(a\) and \(b\) be distinct points in $\mathbf{R}^n$.
Show that the set of all points that are closer (in Euclidean norm) to \(a\) than \(b\), i.e., $\{x \, | \, \| x - a \|_2 \leq \| x - b \|_2\}$,
is a halfspace. Describe it explicitly as an inequality of the form $c^T x \leq d$. Draw a picture.
\begin{Solution}
	Let us take a point between \(a\) and \(b\):
	\[
		x_0 \;=\; \frac{a+b}{2}.
	\]
	Then let \(c\) be the vector pointing from \(a\) to \(b\):
	\[
		c \;=\; b - a.
	\]
	If we move in a direction perpendicular to \(c\), we neither move closer to \(a\) nor to \(b\). Thus, we can describe a hyperplane defining the boundary between the two halfspaces:
	\[
		\{\,x : c^T (x - x_0) = 0\}.
	\]
	Therefore, the halfspace consisting of points closer to \(a\) is
	\[
		\{\,x : c^T x \;\le\; c^T x_0\}.
	\]
	(A figure may be provided later.)
\end{Solution}



\vspace{0.15in}

\subsection*{Exercise 2.8}
Which of the following sets \(S\) are polyhedra? If possible, express \(S\) in the form $S = \{x \, | \, Ax \preceq b , \, Fx = g\}$.
\vspace{0.1in}
\begin{enumerate}[label=(\alph*)]

	\item
	      \[
		      S = \{\,y_1 a_1 + y_2 a_2 \mid -1 \leq y_1 \leq 1,\,-1 \leq y_2 \leq 1\},
		      \quad
		      \text{where }a_1, a_2 \in \mathbf{R}^n.
	      \]

	      \begin{Solution}
		      Let us look at the boundary, where \(y_1\) and \(y_2\) are equal to their constraints:
		      \[
			      C
			      =
			      \{\,a_1 + a_2,\,-a_1 + a_2,\,
			      a_1 - a_2,\,-a_1 - a_2\}.
		      \]
		      We see the swept region is a kind of parallelogram, which is a polyhedron.
	      \end{Solution}

	\item
	      \[
		      S
		      =
		      \{\,x \in \mathbf{R}^n \mid x \succeq 0,\,
		      \mathbf{1}^T x = 1,\,
		      \sum_{i=1}^n x_i a_i = b_1,\,
		      \sum_{i=1}^n x_i a_i^2 = b_2\},
	      \]
	      where \(a_1,\dots,a_n \in \mathbf{R}\) and \(b_1, b_2 \in \mathbf{R}\).

	      \begin{Solution}
		      We see that it is a polyhedron by writing it as
		      \[
			      -\,\mathbf{I}\,x \preceq 0
		      \]
		      and
		      \[
			      \begin{bmatrix}
				      1     & 1     & \cdots & 1     \\
				      a_1   & a_2   & \cdots & a_n   \\
				      a_1^2 & a_2^2 & \cdots & a_n^2
			      \end{bmatrix}
			      x
			      =
			      \begin{bmatrix}
				      1   \\[6pt]
				      b_1 \\[6pt]
				      b_2
			      \end{bmatrix}.
		      \]
	      \end{Solution}

	\item
	      \[
		      S
		      =
		      \bigl\{\,
		      x \in \mathbf{R}^n
		      \,\bigm|\,
		      x \succeq 0,\,
		      x^T y \leq 1
		      \text{ for all }y\text{ with }\|y\|_2 = 1
		      \bigr\}.
	      \]

	      \begin{Solution}
		      The term \(x^T y\) is the dot product with a unit vector. Since \(x\) is chosen arbitrarily, we can set
		      \(\displaystyle y = \frac{x}{\|x\|_2}\)
		      to examine the boundary of the set, which gives
		      \[
			      \|x\|_2 \,\leq\, 1.
		      \]
		      Along with \(x \succeq 0\), this describes a portion of the \(n\)-dimensional sphere lying in the nonnegative orthant. Hence, it is not a polyhedron.
	      \end{Solution}

	\item
	      \[
		      S
		      =
		      \bigl\{\,
		      x \in \mathbf{R}^n
		      \,\bigm|\,
		      x \succeq 0,\,
		      x^T y \leq 1
		      \text{ for all }y\text{ with }
		      \sum_{i=1}^n |y_i| = 1
		      \bigr\}.
	      \]

	      \begin{Solution}
		      Let us again look at the boundary. The quantity \(x^T y\) is largest when \(y_i = 1\) for the coordinate \(i\) that maximizes \(x_i\). Thus the shape is bounded by
		      \(\max_i x_i \leq 1\)
		      and must lie in the nonnegative orthant.
		      Hence, it can be described by
		      \[
			      \begin{bmatrix}
				      \mathbf{I}_n \\
				      -\,\mathbf{I}_n
			      \end{bmatrix}
			      x
			      \preceq
			      \begin{bmatrix}
				      1 \\[3pt]
				      0
			      \end{bmatrix},
		      \]
		      so it is a polyhedron.
	      \end{Solution}

\end{enumerate}


\vspace{0.15in}

\subsection*{Exercise 2.11}
\textit{Hyperbolic sets}. Show that the \textit{hyperbolic} set
\(\{\, x \in \mathbf{R}^2_+ \mid x_1 x_2 \geq 1\}\)
is convex. More generally, show that
\(\{\, x \in \mathbf{R}^n_+ \mid \prod_{i=1}^n x_i \geq 1 \}\)
is convex.

\textit{Hint}. If \(a,b \geq 0\) and \(0 \leq \theta \leq 1\), then
\(a^{\theta} b^{1-\theta} \leq \theta a + (1 - \theta)b\).

\begin{Solution}
	Let us go directly to the general case. Suppose \(x,\,y \succeq 0\) lie in the set, which means:
	\[
		\prod_{i=1}^n x_i \;\geq\; 1
		\quad\text{and}\quad
		\prod_{i=1}^n y_i \;\geq\; 1.
	\]
	We want to show that for any convex combination \(z = \theta x + (1-\theta)\,y\), with \(0 \leq \theta \leq 1\), the point \(z\) also lies in the set; in other words, its components’ product is at least 1:
	\[
		\prod_{i=1}^n z_i
		\;=\;
		\prod_{i=1}^n
		\bigl(\theta x_i + (1-\theta)\,y_i\bigr)
		\;\geq\;
		\prod_{i=1}^n x_i^{\theta}\,y_i^{1-\theta}
		\;\geq\;
		1^{\theta}\,1^{1-\theta}
		\;\geq\;
		1.
	\]
	The first inequality uses the hint
	\(\bigl(\theta a + (1-\theta)\,b\bigr) \geq a^{\theta} b^{1-\theta}\)
	for \(a,b \geq 0\) and \(0 \leq \theta \leq 1\).
	Thus, \(\prod_{i=1}^n z_i \geq 1\), so \(z\) is in the set.
\end{Solution}


\vspace{0.15in}

\subsection*{Exercise 2.12}
Which of the following sets are convex?
\begin{enumerate}[label=(\alph*)]

	\item
	      A slab, \emph{i.e.}, a set of the form
	      \(\{\,x \in \mathbf{R}^n \mid \alpha \leq a^T x \leq \beta\}\).
	      \begin{Solution}
		      Suppose \(x_1\) and \(x_2\) lie in the set. For \(0 \leq \theta \leq 1\), define
		      \[
			      x_3 \;=\; \theta x_1 \;+\; (1-\theta)\,x_2.
		      \]
		      Then
		      \[
			      \alpha
			      \;\le\;
			      \theta\,a^T x_1 + (1-\theta)\,a^T x_2
			      \;\le\;
			      \beta,
		      \]
		      which shows \(x_3\) is also in the set. Therefore, the set is convex.
	      \end{Solution}

	\item
	      A rectangle, \emph{i.e.}, a set of the form
	      \(\{\,x \in \mathbf{R}^n \mid \alpha_i \leq x_i \leq \beta_i,\; i=1,\dots,n\}\).
	      (A rectangle is sometimes called a \emph{hyperrectangle} when \(n>2\).)
	      \begin{Solution}
		      As in the previous example, let \(x,y\) be in the set. Then for \(0 \le \theta \le 1\),
		      \[
			      \alpha_i
			      \;\le\;
			      \theta\,x_i + (1-\theta)\,y_i
			      \;\le\;
			      \beta_i
			      \quad
			      \text{for each }i.
		      \]
		      Hence the rectangle is convex.
	      \end{Solution}

	\item
	      A \emph{wedge}, \emph{i.e.},
	      \(\{\,x \in \mathbf{R}^n \mid a_1^T x \le b_1,\; a_2^T x \le b_2\}\).
	      \begin{Solution}
		      Let \(x_1\) and \(x_2\) lie in the set, and let \(0 \le \theta \le 1\). Then
		      \[
			      a_1^T \bigl(\theta x_1 + (1-\theta)\,x_2\bigr)
			      \;=\;
			      \theta\,a_1^T x_1 + (1-\theta)\,a_1^T x_2
			      \;\le\;
			      \theta\,b_1 + (1-\theta)\,b_1
			      \;=\;
			      b_1.
		      \]
		      Similarly,
		      \[
			      a_2^T \bigl(\theta x_1 + (1-\theta)\,x_2\bigr)
			      \;\le\;
			      b_2.
		      \]
		      Thus the wedge is convex.
	      \end{Solution}

	\item
	      The set of points closer to a given point \(x_0\) than to a given set \(S\),
	      \[
		      \bigl\{
		      x \;\bigm|\;
		      \|x - x_0\|_2 \,\le\, \|x - y\|_2
		      \text{ for all }y \in S
		      \bigr\}.
	      \]
	      \begin{Solution}
		      In the case where \(S\) is just a single point, as shown in an earlier exercise (2.7), the set becomes a halfspace, which is convex. If \(S\) has two points, then we have an intersection of two such halfspaces, which remains convex. By extension, for an arbitrary set \(S\), the set in question is the intersection of halfspaces (one for each point in \(S\)), and since the intersection of convex sets is convex, the result follows.
	      \end{Solution}

	\item
	      The set of points closer to one set \(S\) than another set \(T\), \emph{i.e.},
	      \[
		      \bigl\{
		      x \;\bigm|\;
		      \mathbf{dist}(x,S)
		      \,\le\,
		      \mathbf{dist}(x,T)
		      \bigr\},
	      \]
	      where \(\mathbf{dist}(x,S) = \inf\,\{\|x-z\|_2 : z \in S\}\).
	      \begin{Solution}
		      Consider \(S = \{-1,\,1\}\) and \(T = \{0\}\) on the real line. The points \(-1\) and \(1\) are each closer to \(S\) than to \(T\), so they lie in the set. However, \(x=0\) is not in the set because it is equally close (and hence not strictly closer) to \(S\) compared to \(T\). The set thus fails to be convex, as it does not contain the midpoint of \(-1\) and \(1\). Therefore, in general, this set need not be convex.
	      \end{Solution}

	\item
	      The set \(\{\,x \mid x + S_2 \subseteq S_1\}\), where \(S_1,S_2 \subseteq \mathbf{R}^n\) and \(S_1\) is convex.
	      \begin{Solution}
		      Let \(x_1\) and \(x_2\) lie in \(\{x : x + S_2 \subseteq S_1\}\), and let \(0 \le \theta \le 1\). For any \(s_0 \in S_2\), we must check if
		      \[
			      \theta x_1 + (1-\theta)\,x_2 \;+\; s_0
			      \;\in\;
			      S_1.
		      \]
		      Since \(x_1 + s_0,\,x_2 + s_0 \in S_1\), there exist points
		      \(s_1 = x_1 + s_0\) and \(s_2 = x_2 + s_0\), both in \(S_1\). By the convexity of \(S_1\),
		      \[
			      \theta\,s_1 + (1-\theta)\,s_2 \;\in\; S_1,
		      \]
		      which shows \(\theta x_1 + (1-\theta)x_2 + s_0 \in S_1\). Hence the set is convex.
	      \end{Solution}

	\item
	      The set of points \(x\) whose distance to \(a\) does not exceed a fixed fraction \(\theta\) of the distance to \(b\), \emph{i.e.},
	      \[
		      \{\,x \mid \|x - a\|_2 \,\le\, \theta\,\|x - b\|_2\},
		      \quad
		      \text{where }a \neq b
		      \text{ and } 0 \le \theta \le 1.
	      \]
	      \begin{Solution}
		      The special cases \(\theta = 0\) (which yields \(\{a\}\)) and \(\theta = 1\) (which yields a halfspace) are straightforward. For \(0 < \theta < 1\), we can square both sides:
		      \[
			      \|x - a\|_2^2
			      \;\le\;
			      \theta^2 \,\|x - b\|_2^2.
		      \]
		      Expanding and rearranging leads to
		      \[
			      (1 - \theta^2)\,x^T x
			      \;-\;
			      2\,(a - \theta^2 b)^T x
			      \;+\;
			      \bigl(\|a\|_2^2 \;-\;\theta^2\,\|b\|_2^2\bigr)
			      \;\le\;
			      0,
		      \]
		      Which can be written in a more suggestive form
		      \[
			      \|x - \frac{a-\theta^2 b}{1-\theta^2} \|_2^2 \leq \frac{\theta^2}{(1-\theta^2)^2} \|a-b\|_2^2
		      \]
		      We see the set is a sphere, hence it is convex.
	      \end{Solution}
\end{enumerate}

\vspace{0.15in}

\subsection*{Exercise 2.15}
\textit{Some sets of probability distributions}. Let \(x\) be a real-valued random variable with \textbf{prob}\((x = a_i) = p_i\), \(i = 1,...,n\), where \(a_1 \leq a_2 \leq ... \leq a_n\). Of course \(p \in \mathbf{R}^n\) lies in the standard probability simplex \(P = \{p \, | \, \mathbf{1}^T p = 1, p \succeq 0\}\).
Which of the following conditions are convex in p? (This is, for which of the following conditions in the set of \(p \in P\) that satisfy the condition convex?)
\begin{enumerate}[label=(\alph*)]
	\item
	      \(\alpha \leq \mathbf{E}f(x) \leq \beta\), where \(\mathbf{E}f(x)\) is the expected value of f(x), \textit{i.e.}, \(\mathbf{E}f(x) = \sum_{i=1}^n p_i f(a_i)\). (The function \(f:\mathbf{R} \rightarrow \mathbf{R}\) is given.)
	      \begin{Solution}
		      As we can easily see the function \(p \rightarrow \sum_{i=1}^n p_i f(a_1)\) is linear. So the constrains are linear inequalities, which is convex as it is intersection of two halfspaces and probability simplex.
	      \end{Solution}

	\item
	      \textbf{prob}\((x \ge \alpha) \leq \beta\).
	      \begin{Solution}
		      As we can notice this is equivalent to linear inequality
		      \[
			      \sum_{\alpha_i \ge \alpha}p_i \leq \beta
		      \]
		      So this set must be convex as well.
	      \end{Solution}

	\item
	      \(\mathbf{E}|x^3| \leq \alpha \mathbf{E} |x|\)
	      \begin{Solution}
		      We can rewrite expression to a linear inequality
		      \[
			      0 \leq \sum_{i=1} p_i(\alpha |a_i| - |a_i^3|)
		      \]
	      \end{Solution}

	\item
	      \(\mathbf{E}x^2 \leq \alpha\).
	      \begin{Solution}
		      Again we have a linear inequality
		      \[
			      \sum_{i=1}^n p_i a_i^2 \geq \alpha
		      \]
	      \end{Solution}

	\item
	      \(\mathbf{E}x^2 \leq \alpha\).
	      \begin{Solution}
		      The same as previous one, a linear inequality
		      \[
			      \sum_{i=1}^n p_i a_i^2 \geq \alpha
		      \]
	      \end{Solution}

	\item
	      \(\mathbf{var}(x) \leq \alpha\), where \(\mathbf{var}(x) = \mathbf{E}(x-\mathbf{E}x)^2\) is the variance of \(x\).
	      \begin{Solution}
		      Expanding the expression for variance \(\mathbf{var}(x) = \mathbf{E}(x^2) - (\mathbf{E}(x))^2\)
		      \[
			      \sum_{i=1}^n p_i a_i^2 - (\sum_{i=1}^n p_i a_i)^2 \leq \alpha
		      \]
		      We can see it is a difference of linear and quadratic function. So we can find an example that it is not a convex function. For instance a simple case \(n = 2\), \(a_1 = 1\), \(a_2 = 0\) and we take two points \(p_1 = (1, 0)\) and \(p_2 = (0, 1)\).
		      For \(p_1\) and \(p_2\) we get \(\mathbf{var}(x) = 0\), but for combination like \(p_3 = (1/2, 1/2)\) we get \(\mathbf{var}(x) = 1/2\), so \(\alpha = 1/3\) do not satisfy the convexity.
	      \end{Solution}
\end{enumerate}

\end{document}
