\documentclass[12pt]{article}

\usepackage[utf8]{inputenc}
\usepackage[T1]{fontenc}
\usepackage{lmodern}
\usepackage{amsmath}
\usepackage{latexsym,amsfonts,amssymb,amsthm,amsmath}
\usepackage{enumitem}
\usepackage{hyperref}
\usepackage{float}
\usepackage{graphicx}
\usepackage{subcaption}
\usepackage{booktabs}
\graphicspath{{./images/}}

\setlength{\parindent}{0in}
\setlength{\oddsidemargin}{0in}
\setlength{\textwidth}{6.5in}
\setlength{\textheight}{8.8in}
\setlength{\topmargin}{0in}
\setlength{\headheight}{18pt}

\newenvironment{Solution}
  {\begin{proof}[\textbf{Solution}]}
  {\end{proof}}

\title{Convex Sets}
\author{Kacper Kłos}

\begin{document}

\maketitle

\vspace{0.4in}

\subsection*{Exercise 2.1}
Let $C \subseteq \mathbf{R}^n$ be a convex set, with $x_1, ... , x_k \in C$, and let $\theta_1, ... , \theta_k \in R$ satisfy $\theta_i \geq 0$,
$\theta_1 + ... + \theta_k = 1$. Show that $\theta_1 x_1 + ... + \theta_k x_k \in C$. (The definition of convexity is that
this holds for $k = 2$; you must show it for arbitrary $k$.) \textit{Hint}. Use induction on $k$.
\begin{Solution}
	We know that
	\[
		\theta_1 + \cdots + \theta_{k-2} \;+\; \tilde{\theta}_{k-1} \;+\; \tilde{\theta}_{k}
		\;=\;
		1
		\quad \text{and} \quad
		\theta_1 x_1 + \cdots + \theta_{k-2} x_{k-2} \;+\; \tilde{\theta}_{k-1} \tilde{x}_{k-1} \;+\; \tilde{\theta}_{k} \tilde{x}_{k}
		\;\in\; C.
	\]
	Next, define
	\[
		\theta_{k-1}
		\;=\;
		\tilde{\theta}_{k-1} + \tilde{\theta}_{k}.
	\]
	From this we see that
	\[
		1
		\;=\;
		\frac{\tilde{\theta}_{k-1} + \tilde{\theta}_{k}}{\theta_{k-1}},
	\]
	so, using the fact that \(C\) is convex, we know that there is a \(x_{k-1}\) such that
	\[
		\tilde{\theta}_{k-1} \tilde{x}_{k-1} + \tilde{\theta}_{k} \tilde{x}_{k}
		\;=\;
		\theta_{k-1} x_{k-1}.
	\]
	Plugging this back into the original equations, we get
	\[
		\theta_1 + \cdots + \theta_{k-1} \;=\; 1
		\quad \text{and} \quad
		\theta_1 x_1 + \cdots + \theta_{k-1} x_{k-1} \;\in\; C,
	\]
	thus reducing two \(x_i\)'s into one. Repeating this procedure recursively leaves us with a single point \(x_0\), which must lie in \(C\) by its convexity.
\end{Solution}

\newpage

\subsection*{Exercise 2.2}
Show that a set is convex if and only if its intersection with any line is convex. Show that
a set is affine if and only if its intersection with any line is affine.

\begin{Solution}
First, let us show that if a set \(C\) is convex, then its intersection with any line \(L\) is also convex.
This follows easily because a line itself is convex, and the intersection of convex sets is convex.
Indeed, if \(x_1, x_2 \in C \cap L\) they must belong to \(C\) and \(L\) separately, then by the convexity of \(C\), for any \(\theta \in [0,1]\),
\[
    \theta x_1 + (1 - \theta) x_2 \in C,
\]
and by the convexity of \(L\),
\[
    \theta x_1 + (1 - \theta) x_2 \in L.
\]
Hence,
\[
    \theta x_1 + (1 - \theta) x_2 \in C \cap L.
\]
Implying convexity of $C \cap L$.

For the other direction, let us take any two points \(x_1\) and \(x_2\) in \(C\). 
Consider the line through \(x_1\) and \(x_2\). 
Since \(C\) intersects every line in a convex set, it follows that this line, intersected with \(C\), contains all points between \(x_1\) and \(x_2\). 
In other words, for all \(\theta \in [0,1]\),
\[
    \theta x_1 + (1 - \theta) x_2 \in C.
\]
Because \(x_1\) and \(x_2\) were chosen arbitrarily, it is true for all $x_1, x_2 \in C$, it follows that \(C\) is convex.

Next, let us turn to the affine case. 
Suppose \(A\) is an affine set and let \(L\) be a line, which is also affine. 
The argument is exactly the same as in the convex case, but now we allow \(\theta \in \mathbf{R}\) (not just \([0,1]\)). 
Thus, the intersection \(A \cap L\) is affine if \(A\) is affine, and the converse holds by an identical reasoning.
\end{Solution}




\end{document}
