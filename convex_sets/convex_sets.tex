\documentclass[12pt]{article}

\usepackage[utf8]{inputenc}
\usepackage[T1]{fontenc}
\usepackage{lmodern}
\usepackage{amsmath}
\usepackage{latexsym,amsfonts,amssymb,amsthm,amsmath}
\usepackage[shortlabels]{enumitem}
\usepackage{hyperref}
\usepackage{float}
\usepackage{graphicx}
\usepackage{subcaption}
\usepackage{booktabs}
\graphicspath{{./images/}}

\setlength{\parindent}{0in}
\setlength{\oddsidemargin}{0in}
\setlength{\textwidth}{6.5in}
\setlength{\textheight}{8.8in}
\setlength{\topmargin}{0in}
\setlength{\headheight}{18pt}

\newenvironment{Solution}
  {\begin{proof}[\textbf{Solution}]}
  {\end{proof}}

\title{Convex Sets}
\author{Kacper Kłos}

\begin{document}

\maketitle

\vspace{0.4in}

\subsection*{Exercise 2.1}
Let $C \subseteq \mathbf{R}^n$ be a convex set, with $x_1, ... , x_k \in C$, and let $\theta_1, ... , \theta_k \in R$ satisfy $\theta_i \geq 0$,
$\theta_1 + ... + \theta_k = 1$. Show that $\theta_1 x_1 + ... + \theta_k x_k \in C$. (The definition of convexity is that
this holds for $k = 2$; you must show it for arbitrary $k$.) \textit{Hint}. Use induction on $k$.
\begin{Solution}
	We know that
	\[
		\theta_1 + \cdots + \theta_{k-2} \;+\; \tilde{\theta}_{k-1} \;+\; \tilde{\theta}_{k}
		\;=\;
		1
		\quad \text{and} \quad
		\theta_1 x_1 + \cdots + \theta_{k-2} x_{k-2} \;+\; \tilde{\theta}_{k-1} \tilde{x}_{k-1} \;+\; \tilde{\theta}_{k} \tilde{x}_{k}
		\;\in\; C.
	\]
	Next, define
	\[
		\theta_{k-1}
		\;=\;
		\tilde{\theta}_{k-1} + \tilde{\theta}_{k}.
	\]
	From this we see that
	\[
		1
		\;=\;
		\frac{\tilde{\theta}_{k-1} + \tilde{\theta}_{k}}{\theta_{k-1}},
	\]
	so, using the fact that \(C\) is convex, we know that there is a \(x_{k-1}\) such that
	\[
		\tilde{\theta}_{k-1} \tilde{x}_{k-1} + \tilde{\theta}_{k} \tilde{x}_{k}
		\;=\;
		\theta_{k-1} x_{k-1}.
	\]
	Plugging this back into the original equations, we get
	\[
		\theta_1 + \cdots + \theta_{k-1} \;=\; 1
		\quad \text{and} \quad
		\theta_1 x_1 + \cdots + \theta_{k-1} x_{k-1} \;\in\; C,
	\]
	thus reducing two \(x_i\)'s into one. Repeating this procedure recursively leaves us with a single point \(x_0\), which must lie in \(C\) by its convexity.
\end{Solution}

\newpage

\subsection*{Exercise 2.2}
Show that a set is convex if and only if its intersection with any line is convex. Show that
a set is affine if and only if its intersection with any line is affine.

\begin{Solution}
	First, let us show that if a set \(C\) is convex, then its intersection with any line \(L\) is also convex.
	This follows easily because a line itself is convex, and the intersection of convex sets is convex.
	Indeed, if \(x_1, x_2 \in C \cap L\) they must belong to \(C\) and \(L\) separately, then by the convexity of \(C\), for any \(\theta \in [0,1]\),
	\[
		\theta x_1 + (1 - \theta) x_2 \in C,
	\]
	and by the convexity of \(L\),
	\[
		\theta x_1 + (1 - \theta) x_2 \in L.
	\]
	Hence,
	\[
		\theta x_1 + (1 - \theta) x_2 \in C \cap L.
	\]
	Implying convexity of $C \cap L$.

	For the other direction, let us take any two points \(x_1\) and \(x_2\) in \(C\).
	Consider the line through \(x_1\) and \(x_2\).
	Since \(C\) intersects every line in a convex set, it follows that this line, intersected with \(C\), contains all points between \(x_1\) and \(x_2\).
	In other words, for all \(\theta \in [0,1]\),
	\[
		\theta x_1 + (1 - \theta) x_2 \in C.
	\]
	Because \(x_1\) and \(x_2\) were chosen arbitrarily, it is true for all $x_1, x_2 \in C$, it follows that \(C\) is convex.

	Next, let us turn to the affine case.
	Suppose \(A\) is an affine set and let \(L\) be a line, which is also affine.
	The argument is exactly the same as in the convex case, but now we allow \(\theta \in \mathbf{R}\) (not just \([0,1]\)).
	Thus, the intersection \(A \cap L\) is affine if \(A\) is affine, and the converse holds by an identical reasoning.
\end{Solution}


\vspace{0.15in}

\subsection*{Exercise 2.5}
What is the distance between two parallel hyperplanes
\(\{x \in \mathbf{R}^n \mid a^T x = b_1\}\)
and
\(\{x \in \mathbf{R}^n \mid a^T x = b_2\}\)?

\begin{Solution}
	We know that \(a\) is parallel to the plane. Subtracting both equations, we get
	\[
		a^T (x_1 - x_2) = b_1 - b_2.
	\]
	Taking the Euclidean norm on both sides, and noting that the distance is the minimum possible value of \(\|x_1 - x_2\|_2\), we write
	\[
		\|a^T (x_1 - x_2)\|_2 = |b_1 - b_2|.
	\]
	From the standard norm inequality,
	\[
		\|a^T (x_1 - x_2)\|_2
		\;\le\;
		\|x_1 - x_2\|_2 \,\|a^T\|_2.
	\]
	The equality here represents the minimal distance. Consequently, dividing by constant \(\|a^T\|_2\)
	\[
		\min\bigl(\|x_1 - x_2\|_2\bigr)
		\;=\;
		\frac{|b_1 - b_2|}{\|a^T\|_2}.
	\]
	This value is the distance between the two hyperplanes.
\end{Solution}


\vspace{0.15in}

\subsection*{Exercise 2.7}
\textit{Voronoi description of halfspace}. Let \(a\) and \(b\) be distinct points in $\mathbf{R}^n$.
Show that the set of all points that are closer (in Euclidean norm) to \(a\) than \(b\), i.e., $\{x \, | \, \| x - a \|_2 \leq \| x - b \|_2\}$,
is a halfspace. Describe it explicitly as an inequality of the form $c^T x \leq d$. Draw a picture.
\begin{Solution}
	Let us take a point between \(a\) and \(b\):
	\[
		x_0 \;=\; \frac{a+b}{2}.
	\]
	Then let \(c\) be the vector pointing from \(a\) to \(b\):
	\[
		c \;=\; b - a.
	\]
	If we move in a direction perpendicular to \(c\), we neither move closer to \(a\) nor to \(b\). Thus, we can describe a hyperplane defining the boundary between the two halfspaces:
	\[
		\{\,x : c^T (x - x_0) = 0\}.
	\]
	Therefore, the halfspace consisting of points closer to \(a\) is
	\[
		\{\,x : c^T x \;\le\; c^T x_0\}.
	\]
	(A figure may be provided later.)
\end{Solution}



\vspace{0.15in}

\subsection*{Exercise 2.8}
Which of the following sets \(S\) are polyhedra? If possible, express \(S\) in the form $S = \{x \, | \, Ax \preceq b , \, Fx = g\}$.
\vspace{0.1in}
\begin{enumerate}[label=(\alph*)]

	\item
	      \[
		      S = \{\,y_1 a_1 + y_2 a_2 \mid -1 \leq y_1 \leq 1,\,-1 \leq y_2 \leq 1\},
		      \quad
		      \text{where }a_1, a_2 \in \mathbf{R}^n.
	      \]

	      \begin{Solution}
		      Let us look at the boundary, where \(y_1\) and \(y_2\) are equal to their constraints:
		      \[
			      C
			      =
			      \{\,a_1 + a_2,\,-a_1 + a_2,\,
			      a_1 - a_2,\,-a_1 - a_2\}.
		      \]
		      We see the swept region is a kind of parallelogram, which is a polyhedron.
	      \end{Solution}

	\item
	      \[
		      S
		      =
		      \{\,x \in \mathbf{R}^n \mid x \succeq 0,\,
		      \mathbf{1}^T x = 1,\,
		      \sum_{i=1}^n x_i a_i = b_1,\,
		      \sum_{i=1}^n x_i a_i^2 = b_2\},
	      \]
	      where \(a_1,\dots,a_n \in \mathbf{R}\) and \(b_1, b_2 \in \mathbf{R}\).

	      \begin{Solution}
		      We see that it is a polyhedron by writing it as
		      \[
			      -\,\mathbf{I}\,x \preceq 0
		      \]
		      and
		      \[
			      \begin{bmatrix}
				      1     & 1     & \cdots & 1     \\
				      a_1   & a_2   & \cdots & a_n   \\
				      a_1^2 & a_2^2 & \cdots & a_n^2
			      \end{bmatrix}
			      x
			      =
			      \begin{bmatrix}
				      1   \\[6pt]
				      b_1 \\[6pt]
				      b_2
			      \end{bmatrix}.
		      \]
	      \end{Solution}

	\item
	      \[
		      S
		      =
		      \bigl\{\,
		      x \in \mathbf{R}^n
		      \,\bigm|\,
		      x \succeq 0,\,
		      x^T y \leq 1
		      \text{ for all }y\text{ with }\|y\|_2 = 1
		      \bigr\}.
	      \]

	      \begin{Solution}
		      The term \(x^T y\) is the dot product with a unit vector. Since \(x\) is chosen arbitrarily, we can set
		      \(\displaystyle y = \frac{x}{\|x\|_2}\)
		      to examine the boundary of the set, which gives
		      \[
			      \|x\|_2 \,\leq\, 1.
		      \]
		      Along with \(x \succeq 0\), this describes a portion of the \(n\)-dimensional sphere lying in the nonnegative orthant. Hence, it is not a polyhedron.
	      \end{Solution}

	\item
	      \[
		      S
		      =
		      \bigl\{\,
		      x \in \mathbf{R}^n
		      \,\bigm|\,
		      x \succeq 0,\,
		      x^T y \leq 1
		      \text{ for all }y\text{ with }
		      \sum_{i=1}^n |y_i| = 1
		      \bigr\}.
	      \]

	      \begin{Solution}
		      Let us again look at the boundary. The quantity \(x^T y\) is largest when \(y_i = 1\) for the coordinate \(i\) that maximizes \(x_i\). Thus the shape is bounded by
		      \(\max_i x_i \leq 1\)
		      and must lie in the nonnegative orthant.
		      Hence, it can be described by
		      \[
			      \begin{bmatrix}
				      \mathbf{I}_n \\
				      -\,\mathbf{I}_n
			      \end{bmatrix}
			      x
			      \preceq
			      \begin{bmatrix}
				      1 \\[3pt]
				      0
			      \end{bmatrix},
		      \]
		      so it is a polyhedron.
	      \end{Solution}

\end{enumerate}



\end{document}
